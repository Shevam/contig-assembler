\documentclass[10pt,letterpaper]{article}
\usepackage{amsmath}
\usepackage{fullpage}
\usepackage{enumerate}
\usepackage{fancyhdr}


\setlength{\parindent}{0.0in}
\setlength{\parskip}{0.1in}

% Edit these as appropriate
\newcommand\course{CSCI1820}
\newcommand\semester{Spring 2013}       % <-- current semester
\newcommand\hwnum{5 Write-up}                  % <-- homework number
\newcommand\yourname{Stephanie DeMane} % <-- your name
\newcommand\login{sdemane}           % <-- your CS login



\pagestyle{fancy}
\headheight 28pt
\fancyhead[L]{\yourname, \login\\ \today}
\fancyhead[R]{\course, \semester\\ Homework \hwnum}
\headsep 5px

\begin{document}

My completed code and data is contained in the folder \texttt{assembler/}.

\section{}
The source code is contained in \texttt{src}. The program can be compiled by running \texttt{ant} and run with \texttt{bin/assembler <args>}. I set up a script \texttt{run\_trials} which I used to easily playing with the parameters on my runs.

\section{}
Results from the small first-week genome are contained in \texttt{reduced\_genome/}.

\section{}
Results from the second week reads are contained in the \texttt{full\_genome} folders and labeled by coverage.

\section{}
The approximate value of parameter $a$ (coverage) required to get $c$ contigs with the parameters $l = 50$ and $G = 10000$ is calculated as follows:
\begin{align*}
 \intertext{Given}
 c &= Ne^\frac{Nl}{G}\\
 a &= \frac{Nl}{G}\\
 \intertext{we find}
 c &= \frac{G}{l}ae^{-a}\\
 ae^{-a} &= \frac{cl}{G}\\
 \intertext{which cannot be easily simplified.}
\end{align*}

\section{}
My strategy did not change significantly from my first week algorithmic plan. I added error correction strategies such as removing extraneous tip sequences and merging similar sequences. For the tip removal I followed the algorithm given in the Velvet paper as closely as I could manage. The merging of similar sequences was trickier. I worked within the constraint that the errors in the sequences would only be point substitutions, not indels or more complex errors. Therefore, to create contigs while correcting for mid-sequence point substitutions,
\begin{enumerate}
 \item Find a block with no incoming arcs and create a path containing that block. Create an empty list of working paths. If there are no such blocks, continue to (9).
 \item If there is only one outgoing arc, add the next block to the path and put the path back into the list.
 \item If there is more than one outgoing arc, create multiple copies of the path and add a different block to the end of each copy. Add all of the new paths to the list of paths.
 \item If there are no outgoing arcs, store the path in a final list of complete paths.
 \item Pull the shortest length path from the list of paths. If the working list of paths is empty, go to (1).
 \item If there is another path of the same length as the current path in the working list of paths, compare it to the current path. Otherwise, return to (2).
 \item If the two paths share a final block and are within a certain edit distance of each other, merge the paths. To merge paths, examine the location where they fork. Retain the more likely path -- the path with the greater arc multiplicity at the fork location -- as the current path. Discard the other path, and set aside the blocks which are not shared with the most likely path.
 \item Return to (6).
 \item Examine the list of complete paths. Pull the longest length path and add it to the final set of contigs.
 \item Find the path in list of complete paths which contains the most sequence not already contained in the paths in the final set of contigs.
 \item If the length of new sequence is greater than a certain number (I used $2k$), add it to the set of contigs and return to (10). Otherwise, continue.
 \item Examine the set of blocks set aside while merging paths in (7). If a block is not contained in any of the final contigs, remove it from the master set of blocks.
\end{enumerate}



When running my assembler on the reads provided, I observed that the number of contigs increased as the coverage increased.






\end{document}

